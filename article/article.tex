\documentclass[12pt]{article}
\usepackage{mathptmx}
\usepackage{float}
\usepackage[utf8]{inputenc}  
\usepackage{fix-cm}  % this package allows large \fontsize
\usepackage{tikz}    % this is for graphics. e.g. rectangle on title page
\usepackage{amsmath} % Used by equations
\usepackage[bindingoffset=0.625in,
            left=.5in, right=.5in,
            top=.8125in, bottom=.9375in,
            paperwidth=6.375in, paperheight=9.25in]{geometry}
\begin{document}
As one of the main decomposers in nature, fungi play an important role in our life and maintaining the balance of ecosystem. Fungi are distributed in most ecosystems on the earth. They prefer humid and warm environment and are very sensitive to environmental changes. Due to their multi-functional metabolism, fungi can decompose organic matter and release carbon and inorganic salts into the natural environment, making them re-enter the atmosphere, water and water, and re-enter the material cycle of the ecosystem.

Studies have shown that under certain climatic conditions, the stable flora will decompose *\% of the woody fiber they live on in a short period of time.If decomposers such as fungi and bacteria do not metabolize them back to the environment, they will combine with decaying biological remains, which in turn affects the living environment of fungi.

The growth of fungi has strict requirements on temperature and humidity. For Phlebia\_acerina\_DR60\_A8A, the suitable temperature and humidity range form 1\-2. Beyond this range, the propagation speed of fungi decreases, which is reflected in the fact that the time from 0 to steady state becomes longer.

According to the research, the drastic change of climate will lead to the change of local fungal biomass and even irreversible attack. For a given climate condition, when the temperature difference is (temperature difference), the biomass of fungal community decreases by *\% and the decomposition time of half sawdust increases by *\%
At the same time (other changes, such as changes in the proportion of the population, whether the vulnerable population is superior in the unstable state, etc.)

The increase of species diversity in fungal community is helpful to increase the tolerance of the whole population to environmental changes. Under the given climate conditions, when the number of fungal population increased from a to B, the sensitivity coefficient decreased by *\%, which was shown by the given disturbance (the performance of community B and community a)
\end{document}