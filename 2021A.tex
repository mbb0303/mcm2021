\documentclass[12pt]{article}
\usepackage[2120943]{easymcm}
\usepackage{mathptmx}
\usepackage{float}
\usepackage{subfigure}

\problem{A}
\title{Analysis and Prediction of Fungi Decomposition Process}

\begin{document}

	\begin{abstract}
		Fungi play an indespensable role in the carbon cycle of our ecosystems. They free the trapped carbon in the debri of living creatures out into the cycle through metabolism. Environment for fungi's well-living is delicate. Any turbulence of temoerature, humidity, and the competition from within the populations would bring pressure to the fungal reproducction.
		
		First of all, we assume that the only resources, the woody fibres, are infinite, thus build a mathematical model using N-species CLV equations to describe the growth of fungi populationj without limit of population capacity. To build the model we assume some of the paramaters of the growth of fungi. The model demostrates that.
		
		In the second step, we take the limitation of woody fibres into account, and adjust Model1 to fit the new population model. The new model showa that 123.

		
		
		In addition, we test the minggandu and robustness\\
		\vspace{5pt}
		\textbf{Keywords}:
	\end{abstract}
	
	\maketitle
	\tableofcontents
	\section{Introduction}
	\subsection{Problem Background}
	There are more than 12k kinds of fungi living on earth.As an functionally critical component of our terrestrial ecosystems, fungi free the carbon and other elements out from remains and debris and drive them into the ecosystem circculation. Fungi tend to live in warm and humid environment,and are sensitive with the smallest changes. Their decompose ability varys under different temperature and moisture and different species represents different traits like moisture tolerance, temperature tolerance, competitive rank and so on.

	In this study, we focus on the interaction between different population of fungi and how they interact with microenvironment around them on woddy fibres. We use cmpetitive Lotka-Voterra model to demostrate the competition among different types of fungi.
	
\begin{figure}[htbp]
\centering
\subfigure[Phlebia centrifuga]{
\begin{minipage}[t]{0.3\linewidth}
\centering
\includegraphics[width=1in]{Phlebia_centrifuga.jpg}
\end{minipage}%
}%
\subfigure[Phlebia radiata]{
\begin{minipage}[t]{0.3\linewidth}
\centering
\includegraphics[width=1in]{Phlebia_radiata.jpg}
\end{minipage}%
}%
\subfigure[Hyphodontia arguta]{
\begin{minipage}[t]{0.3\linewidth}
\centering
\includegraphics[width=1in]{Hyphodontia_arguta.jpg}
\end{minipage}
}%
\centering
\caption{Different Fungi}
\end{figure}

	
	\subsection{Restatement of the Problem}
	\begin{itemize}
		\item Build a mathematical model to simulate fungi's degradation process.
		\item Incoporate the interactions between different species to the previous model.
		\item Examine the sensitivity to rapid fluctuations in the environment (e.g., temperature and humidity).
		\item Analyze the advantage and disadvantage for each species under different environment
		\item Examine the influence of biodiversity on fungi population, for instance the resistance against temperature fluctuations.
	\end{itemize}
	
	
	\subsection{Our Work}
	The topic requires us to simulate the growth of fungi community and then consider their impact on the plant material. Our work mainly includes the following:
	\begin{itemize}
		\item Based on the Competitive Lotka-Volterra equations and the data of fungi growth rate with respect to temperature and moisture, a population growth model is established.
		\item Consider the impact of plant material to fungi community to build the decomposition model.
		\item Change the climate and make rapid fluctuations to the model to discuss the outcome.
	\end{itemize}	
	\section{Model Preperation}
	\subsection{Assumptions}
	\begin{itemize}
		\item Fungi community is applied to Competitive Lotka-Volterra equations
		\item We assume the research data is accurate
		\item The parameter estimated from the data, which were generated from standardized experiment, are fungi's inherent traits
	\end{itemize}
	\subsection{Notations}
	Important notations used in this paper are listed in Table \ref{tb:notation}
	\begin{table}[H]
		\begin{center}
		\caption{Notations}
		\begin{tabular}{ccc}
			\toprule
			Symbol& Definition &Unit\\
			\midrule
			$T$&Temperature in Celcius&$^{\circ}C$\\
			\specialrule{0em}{1pt}{1pt}
			$M$&Moisture&MPa\\  
			\specialrule{0em}{1pt}{1pt}
			$n_i$&Fungi's Competitive Rank&$n \text{ is scaled to } [0,1]$\\
			\specialrule{0em}{1pt}{1pt}
			$\rho_{\text{extension}}$&Fungi's Extension Rate&$\text{mm/day}$\\
			\specialrule{0em}{1pt}{1pt}
			$S_i$ &Fungi's Popluation Size& $\mu \text g$\\ 
			\specialrule{0em}{1pt}{1pt}
			$K_i$ & Bioligical Capacity& $\mu \text g$\\
			\specialrule{0em}{1pt}{1pt}
			$M_{\text{woody}}$ & Weight of the Woody&$\mu \text g$\\
			\bottomrule
		\end{tabular}\label{tb:notation}
		\end{center}
	\end{table}		
	\subsection{Data Collection}
	The data we use mainly include several kinds of fungi's growth rate with temperature and moisture and their inherent traits. The data sources are summarized in Table \ref{tb:data}.
	\begin{table}[!htbp]
	\begin{center}
		\caption{Data Sources}
		\begin{tabular}{cll}
			\toprule
			\multicolumn{1}{m{5cm}}{\centering Data}
			&\multicolumn{1}{m{5cm}}{\centering Source}\\
			\midrule
			Fungi Variable and Description & https://www.pnas.org/content/pnas/117/21/11551.full.pdf\\
			Fungal\_trait\_data.csv&https://github.com/dsmaynard/fungal\_biogeography\\
			Fungi\_moisture\_curves.csv& https://github.com/dsmaynard/fungal\_biogeography\\
			Fungi\_temperature\_curves.csv& https://github.com/dsmaynard/fungal\_biogeography\\
			\bottomrule
		\end{tabular}\label{tb:data}
	\end{center}	
	\end{table}
	\section{Model 1:Fungi Population Prediction Model}
		\subsection{Competitive Lotca-Voterra Equations}
        We assume that there is only one kind of fungi, Phlebia acerina DR60 A8A,and using data as follows:
\begin{align}%单位改正体,下同
    T=22^{\circ}C,M=-0.5\text{MPa},n=0.97,V_{ex}=8.51\text{mm/day},\nonumber\\
    \rho_{hy}=0.27\mu g/cm^2,v_{de}=73.39\pm 10.22\%/122\text{days}\nonumber
\end{align}



 where r is pamrameter of describing the speed of growth, and k is bioligical capacity.

We assume that there are N kinds of different fungi, and the biomass of each kind of fungi is $S_i$. We only concern about the relative biomass of each kind of fungi, so we substitute $S_i$ with $x_i=S_i/K_i$.

We use the Competitive Lotca-Voterre equations, a model for the population dynamics of species competing for some resource. 

\begin{align}
    \frac{dS_i}{dt}=r_i S_i  (1- \frac{\sum_{j=1}^{N}\alpha_{ij}S_j}{K_i})
\end{align}

Using the substitution $x_i=S_i/K_i$, the equations can be written as below:

\begin{align}
    \frac{dx_i}{dt}=r_i x_i  (1- \frac{\sum_{j=1}^{N}\alpha_{ij}x_jK_j}{K_i})
\end{align}

where
\begin{align}
    r_i&=\frac{v_{extension}}{R}\\
    K_i&=C\rho_i,C=const\\
    \alpha_{ij}&=\exp{1-\frac{n_i}{n_j}}\\
    \epsilon&=0.33
\end{align}

$\epsilon\:$is efficiency, according to the reference[3].
		\subsection{Parameter Estimation of LV Equations}
        We assume that $\alpha_{ij}=\exp(1-n_i/n_j)$, making the fungi with larger n having advantages over fungi with smaller n.
			\subsubsection{Carrying Capacity:K}
			\subsubsection{Inherent Per-capita Growth Rate:r}
            When there's only one kind of fungi, it evolve as he equation below:

            \begin{align}
                \frac{dx(t)}{dt}=r(T,M)x(t)(1-\frac{S(t)}{K(T,M)})
            \end{align}
            So we assume that $r=\left< \frac{dS}{dt} \right>$
		\subsection{Results}
        We choose several kinds of fungi, the data are shown as follow:

and the result after enough long time of evolution is shown in Figure 1, where temperature $T=25^{\circ}C$, moisture $M=-0.5MPa$

\begin{figure}[H]
    \centering
    \includegraphics[width=.6\textwidth]{25_05.png}
    \caption{The result of Model 1}\label{fig:result1}%引用标记
    \end{figure}

%---------------------------------------------			
	\subsection{Detail 1 about Model 1}

%-------------------------------------------------








	\section{Model 2:Woody Fibres Decomposition Model}
	\subsection{Decomposition Equations}
	\subsection{Single Population}
		\subsubsection{Population Equation}
		According to the reference\cite{2}, the decomposition of wood confirms to the following model:

\begin{align}
    \frac{dS}{dt}=rS(1-S/K)\times M_{woody}\\
    \frac{dM_{woody}}{dt}=-\frac{r}{\epsilon}SM_{woody}
\end{align}
		\subsubsection{Results}
	\subsection{Multi Population}
		\subsubsection{Population Equation}
We assume that there are N kinds of different fungi, and the biomass of each kind of fungi is $S_i$. We only concern about the relative biomass of each kind of fungi, so we substitute $S_i$ with $x_i=S_i/K_i$.

We use the N-species Competitive Lotca-Voterra equations(\ref{CLV}).Using the substitution $x_i=S_i/K_i$, the equations are shown as follows:

\begin{align}
    \frac{dx_i}{dt}=r_i x_i  (1- \frac{\sum_{j=1}^{N}\alpha_{ij}x_jK_j}{K_i})
\end{align}
the parameters are listed as follows:
\begin{align}
    r_i&=\frac{v_{extension}}{R}\\
    K_i&=C\rho_i,C=const\\
    \alpha_{ij}&=\exp{1-\frac{n_i}{n_j}}
\end{align}

where $\epsilon=0.33$is efficiency, according to\cite{2}.

		\subsubsection{Results}


The results are shown in Figure \ref{fig:result}

\begin{figure}[H]
	\centering
	\includegraphics[width=.6\textwidth]{25_05_de.png}
	\caption{Decomposition Model}\label{fig:result}
\end{figure}

	\section{Test the Models}
\subsection{Sensitivity Analysis}
\subsection{Robustness}
	\section{Conclusion}
\subsection{Summary of Results}
\subsection{Strength}
\subsection{Possible Improvements}
	\section*{Referencese}
\begin{thebibliography}{99}
    \bibitem{1}Nicky Lustenhouwer, Daniel S. Maynard, Mark A. Bradford, Daniel L. Lindner, Brad Oberle,
    Amy E. Zanne, and Thomas W. Crowther, "A trait-based understanding of wood decomposition by
    fungi," Proceedings of the National Academy of Sciences of the United States, May 13, 2020.
    Proceedings of the National Academy of Sciences of the United States, May 13, 2020.
    \bibitem{2}Dang, Christian K. , et al. "Temperature oscillation coupled with fungal community shifts can modulate warming effects on litter decomposition." \emph{Ecology} 90.1(2009).
\end{thebibliography}
	\section*{Appendix}
\begin{table}[H]
		\begin{center}
		\caption{6 kinds of fungi trait}
		\begin{tabular}{ccccc}
			\toprule
			Number&Name &$n$ &Moisture Tolerance&$\rho_{\textrm{hyphae}}$\\
			\midrule
			1&Phlebia\_acerina\_DR60\_A8A&0.9459&1.28&0.27 \\
			\specialrule{0em}{1pt}{1pt}
			2&Tyromyces\_chioneus\_HHB11933\_B10F&0.6486&1.19&0.06\\
			\specialrule{0em}{1pt}{1pt}
			3&Hyphoderma\_setigerum\_HHB12156\_B3H&0.5675&1.19&0.09\\
			\specialrule{0em}{1pt}{1pt}
			4&Hyphodontia\_crustosa\_HHB13392\_B7B&0.3243&1.19&0.12\\
			\specialrule{0em}{1pt}{1pt}
			5&Armillaria\_sinapina\_PR9&0.1351&1.74&0.12\\
			\specialrule{0em}{1pt}{1pt}
			6&Armillaria\_gallica\_EL8\_A6F&0.02702&2.08&1.02\\
			\bottomrule
		\end{tabular}\label{tb:fungitrait}
		\end{center}
\end{table}		

	
\end{document}