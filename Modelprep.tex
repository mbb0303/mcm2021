\section{Model Preperation}
	\subsection{Assumptions}
	\begin{itemize}
		\item Fungi community is applied to Competitive Lotka-Volterra equations
		\item We assume the research data is accurate
		\item The parameter estimated from the data, which were generated from standardized experiment, are fungi's inherent traits
	\end{itemize}
	\subsection{Notations}
	Important notations used in this paper are listed in Table \ref{tb:notation}
	\begin{table}[H]
		\begin{center}
		\caption{Notations}
		\begin{tabular}{ccc}
			\toprule
			Symbol& Definition &Unit\\
			\midrule
			$T$&Temperature in Celcius&$^{\circ}C$\\
			\specialrule{0em}{1pt}{1pt}
			$M$&Moisture&MPa\\  
			\specialrule{0em}{1pt}{1pt}
			$n_i$&Fungi's Competitive Rank&$n \text{ is scaled to } [0,1]$\\
			\specialrule{0em}{1pt}{1pt}
			$\rho_{\text{extension}}$&Fungi's Extension Rate&$\text{mm/day}$\\
			\specialrule{0em}{1pt}{1pt}
			$S_i$ &Fungi's Popluation Size& $\mu \text g$\\ 
			\specialrule{0em}{1pt}{1pt}
			$K_i$ & Bioligical Capacity& $\mu \text g$\\
			\specialrule{0em}{1pt}{1pt}
			$M_{\text{woody}}$ & Weight of the Woody&$\mu \text g$\\
			\bottomrule
		\end{tabular}\label{tb:notation}
		\end{center}
	\end{table}		
	\subsection{Data Collection}
	The data we use mainly include several kinds of fungi's growth rate with temperature and moisture and their inherent traits. The data sources are summarized in Table \ref{tb:data}.
	\begin{table}[!htbp]
	\begin{center}
		\caption{Data Sources}
		\begin{tabular}{cll}
			\toprule
			\multicolumn{1}{m{5cm}}{\centering Data}
			&\multicolumn{1}{m{5cm}}{\centering Source}\\
			\midrule
			Fungi Variable and Description & https://www.pnas.org/content/pnas/117/21/11551.full.pdf\\
			Fungal\_trait\_data.csv&https://github.com/dsmaynard/fungal\_biogeography\\
			Fungi\_moisture\_curves.csv& https://github.com/dsmaynard/fungal\_biogeography\\
			Fungi\_temperature\_curves.csv& https://github.com/dsmaynard/fungal\_biogeography\\
			\bottomrule
		\end{tabular}\label{tb:data}
	\end{center}	
	\end{table}